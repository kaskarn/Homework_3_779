\documentclass[]{article}
\usepackage{lmodern}
\usepackage{amssymb,amsmath}
\usepackage{ifxetex,ifluatex}
\usepackage{fixltx2e} % provides \textsubscript
\ifnum 0\ifxetex 1\fi\ifluatex 1\fi=0 % if pdftex
  \usepackage[T1]{fontenc}
  \usepackage[utf8]{inputenc}
\else % if luatex or xelatex
  \ifxetex
    \usepackage{mathspec}
    \usepackage{xltxtra,xunicode}
  \else
    \usepackage{fontspec}
  \fi
  \defaultfontfeatures{Mapping=tex-text,Scale=MatchLowercase}
  \newcommand{\euro}{€}
\fi
% use upquote if available, for straight quotes in verbatim environments
\IfFileExists{upquote.sty}{\usepackage{upquote}}{}
% use microtype if available
\IfFileExists{microtype.sty}{%
\usepackage{microtype}
\UseMicrotypeSet[protrusion]{basicmath} % disable protrusion for tt fonts
}{}
\usepackage[margin=1in]{geometry}
\usepackage{color}
\usepackage{fancyvrb}
\newcommand{\VerbBar}{|}
\newcommand{\VERB}{\Verb[commandchars=\\\{\}]}
\DefineVerbatimEnvironment{Highlighting}{Verbatim}{commandchars=\\\{\}}
% Add ',fontsize=\small' for more characters per line
\usepackage{framed}
\definecolor{shadecolor}{RGB}{248,248,248}
\newenvironment{Shaded}{\begin{snugshade}}{\end{snugshade}}
\newcommand{\KeywordTok}[1]{\textcolor[rgb]{0.13,0.29,0.53}{\textbf{{#1}}}}
\newcommand{\DataTypeTok}[1]{\textcolor[rgb]{0.13,0.29,0.53}{{#1}}}
\newcommand{\DecValTok}[1]{\textcolor[rgb]{0.00,0.00,0.81}{{#1}}}
\newcommand{\BaseNTok}[1]{\textcolor[rgb]{0.00,0.00,0.81}{{#1}}}
\newcommand{\FloatTok}[1]{\textcolor[rgb]{0.00,0.00,0.81}{{#1}}}
\newcommand{\CharTok}[1]{\textcolor[rgb]{0.31,0.60,0.02}{{#1}}}
\newcommand{\StringTok}[1]{\textcolor[rgb]{0.31,0.60,0.02}{{#1}}}
\newcommand{\CommentTok}[1]{\textcolor[rgb]{0.56,0.35,0.01}{\textit{{#1}}}}
\newcommand{\OtherTok}[1]{\textcolor[rgb]{0.56,0.35,0.01}{{#1}}}
\newcommand{\AlertTok}[1]{\textcolor[rgb]{0.94,0.16,0.16}{{#1}}}
\newcommand{\FunctionTok}[1]{\textcolor[rgb]{0.00,0.00,0.00}{{#1}}}
\newcommand{\RegionMarkerTok}[1]{{#1}}
\newcommand{\ErrorTok}[1]{\textbf{{#1}}}
\newcommand{\NormalTok}[1]{{#1}}
\ifxetex
  \usepackage[setpagesize=false, % page size defined by xetex
              unicode=false, % unicode breaks when used with xetex
              xetex]{hyperref}
\else
  \usepackage[unicode=true]{hyperref}
\fi
\hypersetup{breaklinks=true,
            bookmarks=true,
            pdfauthor={Antoine Baldassari},
            pdftitle={Homework \#3},
            colorlinks=true,
            citecolor=blue,
            urlcolor=blue,
            linkcolor=magenta,
            pdfborder={0 0 0}}
\urlstyle{same}  % don't use monospace font for urls
\setlength{\parindent}{0pt}
\setlength{\parskip}{6pt plus 2pt minus 1pt}
\setlength{\emergencystretch}{3em}  % prevent overfull lines
\setcounter{secnumdepth}{0}

%%% Use protect on footnotes to avoid problems with footnotes in titles
\let\rmarkdownfootnote\footnote%
\def\footnote{\protect\rmarkdownfootnote}

%%% Change title format to be more compact
\usepackage{titling}

% Create subtitle command for use in maketitle
\newcommand{\subtitle}[1]{
  \posttitle{
    \begin{center}\large#1\end{center}
    }
}

\setlength{\droptitle}{-2em}
  \title{Homework \#3}
  \pretitle{\vspace{\droptitle}\centering\huge}
  \posttitle{\par}
  \author{Antoine Baldassari}
  \preauthor{\centering\large\emph}
  \postauthor{\par}
  \predate{\centering\large\emph}
  \postdate{\par}
  \date{November 17, 2015}



\begin{document}

\maketitle


\begin{Shaded}
\begin{Highlighting}[]
\KeywordTok{print}\NormalTok{(}\StringTok{"ok"}\NormalTok{)}
\end{Highlighting}
\end{Shaded}

\begin{verbatim}
## [1] "ok"
\end{verbatim}

\begin{enumerate}
        \item Analyze the arsenic data using a standard conditionally-conjugate specification\\
        
        Let the within- and between- group sampling models be normal with:
        \begin{align*}
            \phi_j &= \left\{ y | \phi_j \right\}, \ p(y|\phi_j) = \text{normal}(\theta_j, \sigma^2) \text{~ (within group) } \\
            \psi &= \left\{ \mu, \tau^2 \right\}, \ p(\theta_j|\psi) = \text{normal}(\mu, \tau^2) \text{~ (between-group) }
        \end{align*}
        In this conditionally-conjugate specification:
        \begin{align*}
            1/\sigma^2 & \sim \text{gamma }(\nu_0/2, \nu_0\sigma^2_0/2)\\
            1/\tau^2 & \sim \text{gamma }(\eta_0/2,\eta_0\tau^2/2)\\
            \mu & \sim \text{normal }(\mu_0, \gamma_0^2)
        \end{align*}
        The full conditional distribution of the parameters can be found to be:
        \begin{align*}
            & \left\{ \theta_j | y_{1,j},\ldots,y_{n_j,j},\sigma^2 \right\} \sim \text{normal }\left( \frac{n_j\bar{y}_j/\sigma^2+\mu/\tau^2}{n_j/\sigma^2 + 1/\tau^2}, \left[ n_j/\sigma^2+1/\tau^2\right]^{-1} \right) \\
            & \left\{\mu | \theta_1,\ldots,\theta_m,\tau \right\} \sim \text{normal } \left( \frac{m\bar{\theta}/\tau^2 + \mu_0/\gamma_0^2}{m/\tau^2+1/\gamma_0^2},\left[ m/\tau^2 + 1/\gamma_0^2\right]^{-1} \right)\\
            & \left\{ 1/\tau^2 | \theta_1, \ldots, \theta_m, \mu \right\} \sim \text{gamma }\left( \frac{\eta_0 +m}{2},\frac{\eta_o\tau^2_0+\sum\left(\theta_j -\mu \right)^2}{2} \right) \\
            & \left\{1/\sigma^2 | \boldsymbol{\theta}, \boldsymbol{y_1}, \ldots, \boldsymbol{y_n} \right\} \sim \text{gamma }\left( \frac{1}{2}\left[ \nu_0 + \sum\limits_{j=1}^m n_j \right], \frac{1}{2}\nu_0\sigma_0^2 + \sum\limits_{j=1}^m \sum\limits_{i=1}^{n_j} \left(y_{i,j} -\theta_j \right)^2 \right)
        \end{align*}
        We pick relatively uninformative priors, centering $\mu$ around $1$ with large variance $\tau^2 = 1000$
        The marginal distributions of $\theta_1, \ldots, \theta_m, \mu, \sigma^2$ and $\tau^2$ can be obtained from the full condition distributions using a Monte-Carlo Markov-Chain algorithm, Gibbs sampling, which follows the procedure (accompanied with R code):
  
        \begin{itemize}
          \item Set prior distributions
          \item Set initial values for algorithm
          \item Create a Markov chain of $S$ items for each parameter:
            \begin{itemize}
              \item ok
            \end{itemize}
          \item Estimate marginal distributions from samples stored in the Markov chains
        \end{itemize}
        We first check that the MCMC model converged for all four statistics:
\end{enumerate}

\end{document}
